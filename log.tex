%log.tex
\section{更新日志}
    当前编译日期为
    \fcolorbox{bupt}{bupt}{\textbf{\textcolor{white}\today}}。
    \paragraph{2020/9/6}模板制作完毕。
    \paragraph{2020/9/20}一二章及其附录更新完毕。
    \paragraph{2020/9/27}为第二章补充一道功率谱计算的例题,修正了附录D中的错误。
    \paragraph{2020/9/28}写完3.1节。
    \paragraph{2020/9/29}开始3.2节。略微调整了第三章的开头,去掉了附录E的公式编号。在第一章里打了广告。
    \paragraph{2020/9/30}结束3.2.2节,决定按照杨老师公开课的顺序编排这一节。
    \paragraph{2020/10/1}结束3.2.4节。宿舍一个人好无聊,希望看到这个笔记的人找我玩\footnote{然而这天并没有在\href{https://github.com/phydx0803/BuptPoCNoteBook/}{GitHub}上更新。}$\ldots$
    \paragraph{2020/10/2}结束3.2节,创造了很多已知的bug。
    \paragraph{2020/10/3}嘴上说着停更但还是修改了3.2.3小小节的错误。
    \paragraph{2020/10/4}导言区新引入了一个宏包{\ttfamily bm}
    \paragraph{2020/10/4}写完了3.3节。
    \paragraph{2020/10/8}写完了3.4节
    \paragraph{2020/10/9}写完第三章,好累!
    \paragraph{2020/10/13}开始更新第四章。
    \paragraph{2020/10/14}发现\appref{appendix:I}最后一节中的一个小错误。
    \paragraph{2020/10/15}更新完DSB-SC AM。
    \paragraph{2020/10/16}解决了交叉链接中仅编号可以专跳但文字不能转跳的问题。
    \paragraph{2020/10/16}写完标准调幅AM。
    \paragraph{2020/10/16}在页眉上新增了转跳点,提升了文章的可阅读性。
    \paragraph{2020/10/19}写完SSB与VSB。
    \paragraph{2020/10/21}写完角度调制抗噪声性能的相关内容。
    \paragraph{2020/10/22}很无聊地微调了封面。
    \paragraph{2020/10/23}开始写角度调制的内容。
    \paragraph{2020/10/24}开始写角度调制的抗噪声性能。
    \paragraph{2020/10/25}写完第四章。
    \paragraph{2020/10/25}咱好开心!\footnote{咱好开心!}
    \paragraph{2020/10/26}开始写第五章。
    \paragraph{2020/10/28}修正了第二章功率谱计算的一个错误。
    \paragraph{2020/10/29}写完5.2小节。
    \paragraph{2020/10/31}补充了一点高斯分布相关函数。
    \paragraph{2020/11/1}修正了前面章节中一些已知的错误。
    \paragraph{2020/11/2}写完5.3匹配滤波器。
    \paragraph{2020/11/3}有小姐姐反馈bug啦好开心。
    \paragraph{2020/11/4}陆陆续续修复了一些已知错误。
    \paragraph{2020/11/8}期中考试歇着。
    \paragraph{2020/11/9}开始写5.4节。
    \paragraph{2020/11/9}怎么有个错别字(半恼。
    \paragraph{2020/11/13}超喜欢杨欣洁小姐姐。
    \paragraph{2020/11/19}开始更新第六章,话说第五章抗噪分析这么大个错咋没人反应?
    \paragraph{2020/11/21}终于赶上进度了。
    \paragraph{2020/12/07}沉迷小姐姐懒得更新。
    