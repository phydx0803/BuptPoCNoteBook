%preset
%=========================================================
%该模板依照是按照作者个人喜好编写的理工类笔记模板
%适合于公式偏多的理工类课程笔记
%由北京邮电大学信息与通信工程学院201821116班2018210779刘淙溪编写
%如有问题可以自行修改或联系QQ1310541573
%重要提示:
%   1. 请确保使用 UTF-8 编码保存
%   2. 请使用 XeLaTeX编译
%   3. 最初编写环境为Tex Live2020
%=========================================================
\usepackage{syntonly}
\usepackage{color,xcolor}
\definecolor{winered}{rgb}{0.5,0,0}
\definecolor{lightlightgray}{rgb}{0.9,0.9,0.9}
\definecolor{olivegreen}{rgb}{0,0.5,0}
\definecolor{pur}{rgb}{0.7,0,0.7}
\definecolor{bupt}{rgb}{0.00392,0.227451,0.537255}
%==================== 电子文档链接 ========================
\usepackage{setspace,hyperref}
\newcommand{\linkcolor}{bupt}
\hypersetup{
    colorlinks=true,
    linkcolor=black,
    citecolor=\linkcolor
            }
\usepackage{pifont}
\newcommand{\firstpageofchapter}{\mbox{\makebox[0pt][l]{$\uparrow$}\rule[1.5ex]{0.5em}{0.4pt}}}
\newcommand{\firstpageofcontent}{\mbox{\makebox[0pt][l]{$\Uparrow$}\rule[1.5ex]{0.6em}{0.3pt}\hspace{-0.6em}\rule[1.7ex]{0.6em}{0.3pt}}}
\newcommand{\HyperBack}[1]{\fancyhead[R]{\hyperref[#1]{\firstpageofchapter}}}
\newcommand{\thmref}[1]{{\texttt{\color{\linkcolor}\hyperref[#1]{定理}}{\thinspace\ref{#1}}}}
\newcommand{\defref}[1]{{\texttt{\color{\linkcolor}\hyperref[#1]{定义}}{\thinspace\ref{#1}}}}
\newcommand{\eqaref}[1]{{\color{\linkcolor}\texttt{\hyperref[#1]{式}}{(\thinspace\ref{#1})}}}
\newcommand{\figref}[1]{{\texttt{\color{\linkcolor}\hyperref[#1]{图}}{\thinspace\ref{#1}}}}
\newcommand{\tabref}[1]{{\texttt{\color{\linkcolor}\hyperref[#1]{表}}{\thinspace\ref{#1}}}}
\newcommand{\appref}[1]{{\texttt{\color{\linkcolor}\hyperref[#1]{附录}}{\thinspace\ref{#1}}}}
\newcommand{\secref}[1]{\texttt{\color{\linkcolor}第\ref{#1}章}}
%==================== 文章字体设定 ========================
\usepackage{fontspec}
%\usepackage{times}
\setmainfont{Times New Roman}  
\setsansfont{Consolas}
\setmonofont{Courier New}
%==================== 数学符号公式 ========================
\usepackage{amsmath}                % AMS LaTeX宏包
\usepackage[style=1]{mdframed}
\usepackage{amsthm}
\usepackage{amsfonts}
\usepackage{mathrsfs}               % 英文花体字 体
\usepackage{amssymb}                     % 数学公式中的黑斜体
\usepackage{bm}
\newcommand{\dif}{\text{\upshape d}}
\newcommand{\degree}{^\circ}          % 角度制度数符号
\usepackage{mathtools}
\newtagform{buptblue}[\color{bupt}]{\color{bupt}(}{\color{bupt})}
\usetagform{buptblue}
\newcommand{\abs}[1]{\left\vert #1 \right\vert}
\usepackage{extarrows}
%=================== 页边距尺寸调整 =======================
\usepackage{geometry}                
\geometry{top=2.5cm,bottom=2.5cm,left=2.5cm,right=2.5cm}
\geometry{headheight=1cm,headsep=1cm,footskip=1cm}
\linespread{1.35}\selectfont
%===================== 按章编号 ===========================
\numberwithin{equation}{subsection}
\renewcommand{\theequation}{\arabic{section}-\arabic{subsection}-\arabic{equation}}
\numberwithin{table}{section}
\renewcommand{\thetable}{\arabic{section}-\arabic{table}}
\numberwithin{figure}{section}
\renewcommand{\thefigure}{\arabic{section}-\arabic{figure}}
\numberwithin{footnote}{subsection}
\renewcommand{\thefootnote}{\ding{\numexpr171+\value{footnote}}}
%================= 页眉页脚 ===============================
\usepackage{fancyhdr}
\fancyhf{}
\chead{\color{\linkcolor}\zihao{-5}\leftmark}
\fancyfoot[C]{\color{\linkcolor}\zihao{5} --\hspace{0.33em}{\thepage}\hspace{0.33em}--}
\fancyhead[L]{\hyperlink{TableOfContents}{\firstpageofcontent}}
\renewcommand{\headrulewidth}{0.4pt}
\fancypagestyle{headings}%章首页
    {
        \fancyhf{}
        \fancyhead[L]{\hyperlink{TableOfContents}{\firstpageofcontent}}
        \fancyfoot[C]{\color{\linkcolor}\zihao{5} --\hspace{0.33em}{\thepage}\hspace{0.33em}--}
        \renewcommand{\headrulewidth}{0pt}
    }
\fancypagestyle{myheadings}%附录页
    {
        \fancyhf{}
        \chead{\color{\linkcolor}\zihao{5}附\hspace{2em}录}
        \fancyfoot[C]{\color{\linkcolor}\zihao{5} --\hspace{0.33em}{\thepage}\hspace{0.33em}--}
        \fancyhead[L]{\hyperlink{TableOfContents}{\firstpageofcontent}}
        \renewcommand{\headrulewidth}{0.4pt} 
    }
%================= 章节格式 ===============================
\usepackage{titlesec}
\usepackage{titletoc}
\renewcommand{\thesection}{\chinese{section}}
\titleformat{\section}[display]
{\normalfont\Large\filcenter\sffamily} 
{
    \clearpage\pagestyle{fancy}
    \thispagestyle{headings}
    \titlerule[1pt]
    \vspace{1pt}
    \titlerule \vspace{1pc}
    \begin{center}
        \color{\linkcolor}
        \LARGE{}第\hspace{0.5em}\thesection\hspace{0.5em}章
    \end{center}
}{-1.6pc}{\LARGE\color{\linkcolor}}
[\vspace{2ex}\color{black}\titlerule\vspace{4ex}]
%subsection
\renewcommand{\thesubsection}{\arabic{section}.\arabic{subsection}}
\titleformat{\subsection}[hang]
{\normalfont\zihao{4}\raggedright\heiti\color{\linkcolor}} 
{\vspace{0.5ex}{\bfseries\thesubsection}}
{0.5em}{}[\vspace{0.2ex}]
%subsubsection
\renewcommand{\thesubsubsection}{\thesubsection.\arabic{subsubsection}}
\titleformat{\subsubsection}[hang]
{\normalfont\normalsize\raggedright\songti\bfseries\color{\linkcolor}} 
{\vspace{0.5ex}\hspace{2em}{\thesubsubsection}}
{0.5em}{}[\vspace{0ex}]
%paragraph 
\renewcommand{\theparagraph}{\arabic{paragraph}}
\CTEXsetup[format={\rmfamily\bfseries\fangsong\zihao{-4}\color{\linkcolor}},name={,.},beforeskip={0em},afterskip={0em},aftername={\hspace{0.2em}},indent=2em]{paragraph}
%subparagrapha
\renewcommand{\thesubparagraph}{\arabic{subparagraph}}
\CTEXsetup[format={\rmfamily\mdseries\songti\zihao{-4}},name={(,)\hspace{-1em}},beforeskip={-1ex},afterskip={-1ex},aftername={\hspace{0em}},indent=2em]{subparagraph}

\setcounter{secnumdepth}{5}
%=======================目录==============================
\setcounter{tocdepth}{3}  
\renewcommand{\contentsname}{目\hspace{2em}录}
\newcommand{\tocafterlength}{10pt}
%section
\titlecontents{section}[0pt]
    {\addvspace{6pt}\filright\bfseries\color{\linkcolor}}
    {\contentspush{\color{\linkcolor}第\thecontentslabel{}章\hspace{\tocafterlength}}}
    {}{\hspace{0.5em}\titlerule*[6pt]{$\cdot$}{\mdseries\rmfamily\normalsize\textcolor{\linkcolor}{\contentspage}}}
    [\addvspace{0em}]
%subsection
\titlecontents{subsection}[22pt]
    {\addvspace{0pt}\filright\mdseries\rmfamily\normalsize}
    {\contentspush{\color{\linkcolor}\thecontentslabel\hspace{\tocafterlength}}}
    {}{\hspace{0.5em}\titlerule*[6pt]{$\cdot$}{\mdseries\rmfamily\normalsize\textcolor{\linkcolor}{\contentspage}}}
    [\addvspace{0em}]
%subsubsection
\titlecontents{subsubsection}[44pt]
    {\addvspace{0pt}\filright\mdseries\rmfamily\normalsize}
    {\contentspush{\color{\linkcolor}\thecontentslabel\hspace{\tocafterlength}}}
    {}{\hspace{0.5em}\titlerule*[6pt]{$\cdot$}{\mdseries\rmfamily\normalsize\textcolor{\linkcolor}{\contentspage}}}
    [\addvspace{0em}]
%====================== 附录 =============================
\usepackage{appendix}
\let\oldappendix=\appendix
\renewcommand{\appendix}
{
    \newpage
    \oldappendix
    \pagestyle{myheadings}
    \renewcommand{\thesection}{\Alph{section}}
    \titlecontents{section}[0pt]
    {\addvspace{6pt}\filright\bfseries\color{\linkcolor}}
    {\contentspush{\color{\linkcolor}附录\thecontentslabel{}\hspace{\tocafterlength}}}
    {}{\hspace{0.5em}\titlerule*[6pt]{$\cdot$}{\mdseries\rmfamily\normalsize\textcolor{\linkcolor}{\contentspage}}}
    [\addvspace{0em}]
    \titleformat{\section}[hang]
    {\clearpage\zihao{-3}\filcenter\heiti\bfseries\color{\linkcolor}} 
    {附录\,{\thesection}}{0.5em}{}[\vspace{2ex}]
    \renewcommand{\theequation}{\thesection-\arabic{subsection}-\arabic{equation}}
    \renewcommand{\thetable}{\thesection-\arabic{table}}
    \renewcommand{\thefigure}{\thesection-\arabic{figure}}
    \renewcommand{\themydefcounter}{\Alph{section}.\arabic{mydefcounter}}
    \renewcommand{\themythmcounter}{\Alph{section}.\arabic{mythmcounter}}
}
%===================== 定理类环境 ==========================
%定理环境
\newcounter{mythmcounter}
\setcounter{mythmcounter}{0}
\numberwithin{mythmcounter}{section}
\newcommand{\mythmindent}{0em}
\renewcommand{\themythmcounter}{\arabic{section}.\arabic{mythmcounter}}
\newcommand{\mythmfont}{\normalsize\rmfamily\bfseries}
\newcommand{\mythmlabel}
{
    \hspace{\mythmindent}
    \fcolorbox{cyan}{cyan}
        {\mythmfont{}定理\hspace{0.5em}\themythmcounter}
}
\newenvironment{mythm}[1]
    {
        \refstepcounter{mythmcounter}
        \par\vspace{0ex}
        \noindent
        \rule[0.3ex]{16cm}{0.8pt}
        \par\noindent\mythmlabel\hspace{0.3em}\mbox{\mythmfont{}#1}
        \kaishu\par
    }
    {\par\noindent\rule[1.5ex]{16cm}{0.8pt}\par\vspace{-0.5ex}}

%定义环境
\newcounter{mydefcounter}
\setcounter{mydefcounter}{0}
\numberwithin{mydefcounter}{section}
\newcommand{\mydefindent}{0em}
\renewcommand{\themydefcounter}{\arabic{section}.\arabic{mydefcounter}}
\newcommand{\mydeffont}{\normalsize\rmfamily\bfseries}
\newcommand{\mydeflabel}
{
    \hspace{\mydefindent}
    \fcolorbox{\linkcolor}{\linkcolor}
        {\mythmfont\textcolor{white}{定义\hspace{0.5em}\themydefcounter}}
}
\newenvironment{mydef}[1]
    {
        \refstepcounter{mydefcounter}
        \par\vspace{0ex}
        \noindent
        \rule[0.3ex]{16cm}{0.8pt}
        \par\noindent\mydeflabel\hspace{0.3em}\mbox{\mydeffont{}#1}
        \kaishu\par
    }
    {\par\noindent\rule[1.5ex]{16cm}{0.8pt}\par\vspace{-0.5ex}}
    
\renewcommand{\emph}[1]{\textcolor{\linkcolor}{#1}}
\newcommand{\Emph}[1]{\colorbox{cyan}{#1}}
\newcommand{\Proof}{\emph{证明:}}
%===================== 列表环境 ==========================
\usepackage{enumitem}
%===================== 抄录环境 ==========================
\usepackage{listings}
\lstset{
    basicstyle=\zihao{5}\sffamily,
    keywordstyle=\sffamily\color{winered},
    commentstyle=\sffamily\color{olivegreen},
    stringstyle=\ttfamily,
    numbers=left, 
    numberstyle=\zihao{-5}\ttfamily,
    breaklines=true,
    frame=tlbr,
    framesep=4pt,
    framerule=0.5pt,
    xleftmargin=2em,
    xrightmargin=2em,
    backgroundcolor=\color{lightlightgray},
    escapeinside=``,
    extendedchars=false,
    columns=flexible,
    captionpos=t,
    lineskip=0pt,
    frameround=ffff
    %mathescape=true
}
%\usepackage{shortvrb}
%\MakeShortVerb{(*}{*)}
%=================== 图表支持宏包 =========================
\usepackage{graphicx}
\usepackage{float}
\usepackage{ifthen}
\usepackage{longtable}
\usepackage{booktabs}
\usepackage{changepage}
\usepackage{enumitem}
\usepackage{varwidth}
\usepackage{array}
\usepackage{colortbl}
\usepackage{subcaption}
\usepackage[all,ps]{xy}
\usepackage[format=hang,font=small,labelfont=normalfont,textfont=it]{caption}
\usepackage{makecell}
\renewcommand{\figurename}{\CJKfamily{zhkai}图}
\renewcommand{\tablename}{\CJKfamily{zhkai}表}
\usepackage{hologo}
\pagestyle{fancy}
\pagenumbering{Roman}