%AppendixI.tex
\section{傅里叶变换及其性质}\label{appendix:I}
\subsection{傅里叶变换的定义}
    \begin{mydef}{傅里叶级数}\label{def:fs}
    任意周期为 $T$ 的周期信号 $s(t)$ 可以展开为傅里叶级数
    \begin{equation}
        \begin{split}
            s(t) &= \sum_{n=-\infty}^{\infty}s_ne^{j\frac{2\pi}{T}nt}\\
            s_n  &= \frac{1}{T}\int_{-\frac{T}{2}}^{\frac{T}{2}}s(t)e^{-j\frac{2\pi}{T}nt}\dif t
        \end{split}
    \end{equation}
    \end{mydef}
    当$T\to\infty$后,离散谱趋向于连续谱,则得到如下的傅里叶变换:
    \begin{mydef}{傅里叶变换}\label{def:ft}
    任意信号\footnote{书中提到的“任意信号”在没有特殊说明下均为有物理意义的信号,无需讨论傅里叶变换的成立条件}
    $x(t)$ 可以进行傅里叶变换
    \begin{equation}
        \begin{split}
            x(t) =\mathscr{F}^{-1} [X(f)]=\int_{-\infty}^{\infty}X(f)e^{j2\pi ft}\dif f\\
            X(f) =\mathscr{F}      [x(t)]=\int_{-\infty}^{\infty}x(t)e^{-j2\pi ft}\dif t
        \end{split}
    \end{equation}
    \end{mydef}
    \subsection{傅里叶变换的性质}
    \paragraph{线性}
    设有$X(f)=\mathscr{F}[x(t)]$,记作$x(t)\leftrightarrow X(f)$,($y(t)\leftrightarrow Y(f)$同理)
        对于任意常数$c_1,c_2$总有
        \begin{equation}
            c_1x(t)+c_2y(t)\leftrightarrow c_1X(f)+c_2Y(f)
        \end{equation}
        \Proof
        \begin{equation*}
            \begin{split}
                \mathscr{F}[c_1x(t)+c_2y(t)] &=\int_{-\infty}^{\infty} [c_1x(t)+c_2y(t)]e^{-j2\pi ft}\dif t\\
                                             &=c_1\int_{-\infty}^{\infty}x(t)e^{-j2\pi ft}\dif t+c_2\int_{-\infty}^{\infty}y(t)e^{-j2\pi ft}\dif t\\
                                             &=c_1X(f)+c_2Y(f)
            \end{split}
        \end{equation*}

    \paragraph{对称性}
        若$x(t)\leftrightarrow X(f)$,则
        \begin{equation}
            X(t)\leftrightarrow x(-f) 
        \end{equation}
        \Proof 根据反变换公式
    \begin{align*}
        x(t) =\int_{-\infty}^{\infty}X(f)e^{j2\pi ft}\dif f\\
        \intertext{做换元$t\rightarrow -t$得}\\
        x(-t) =\int_{-\infty}^{\infty}X(f)e^{-j2\pi ft}\dif f
    \end{align*}
        交换上式中的$t$与$f$得
    \begin{equation*}
        x(-f) =\int_{-\infty}^{\infty}X(t)e^{-j2\pi ft}\dif t
    \end{equation*}
    对照\defref{def:ft}即得$X(t)\leftrightarrow x(-f)$

    \paragraph{尺度变换性质}
        若$x(t)\leftrightarrow X(f)$,对于任意实数$a$有
        \begin{equation}
           x(at)\leftrightarrow \frac{1}{\vert a \vert}X\left(\frac{f}{a}\right)
        \end{equation}
        \Proof

        \subparagraph{}若$a>0$,令$\tau=at$,$t=\tau /a$,$\dif t=\dif \tau /a$,则
        \begin{equation*}
            \mathscr{F}[x(at)]=\frac{1}{a}\int_{-\infty}^{\infty}x(\tau)e^{-j2\pi f\frac{\tau}{a}}\dif \tau=\frac{1}{a}X\left(\frac{f}{a}\right)
        \end{equation*}
        \subparagraph{}若$a<0$,即$a=-\abs{a}$,令$\tau=at$,$t=\tau /a$,$\dif t=\dif \tau /a$,则
        \begin{equation*}
            \mathscr{F}[x(at)]=\frac{-1}{\abs{a}}\int_{-\infty}^{\infty}x(\tau)e^{-j2\pi f\frac{\tau}{a}}\dif \tau=\frac{1}{\abs{a}}X\left(\frac{f}{a}\right)
        \end{equation*}
        
        综上,该性质得证。
    \paragraph{时移性质}
        若$x(t)\leftrightarrow X(f)$,则
        \begin{equation}
            x(t-t_0)\leftrightarrow X(f)e^{-j2\pi ft_0}
        \end{equation}
        \Proof
        \begin{equation*}
            \mathscr{F}[x(t-t_0)]=\int_{-\infty}^{\infty}x(t-t_0)e^{-j2\pi ft}\dif t    
        \end{equation*}
        令$t-t_0=\tau$
        \begin{equation*}
            \mathscr{F}[x(t-t_0)]=\int_{-\infty}^{\infty}x(\tau)e^{-j2\pi f(\tau +t_0)}\dif \tau =X(f)e^{-j2\pi ft_0}   
        \end{equation*}

    \paragraph{频移性质}
        若$x(t)\leftrightarrow X(f)$,则
        \begin{equation}
            x(t)e^{j2\pi f_0t}\leftrightarrow X(f-f_0)
        \end{equation}
        \Proof
        \begin{equation*}
            \mathscr{F}^{-1}[X(f-f_0)]=\int_{-\infty}^{\infty}X(f-f_0)e^{j2\pi ft}\dif f    
        \end{equation*}
        令$f-f_0=\mu$
        \begin{equation*}
            \mathscr{F}^{-1}[X(f-f_0)]=\int_{-\infty}^{\infty}X(\mu)e^{-j2\pi (\mu +f_0)t)}\dif \mu = x(t)e^{j2\pi f_0t} 
        \end{equation*}
    \paragraph{时域微分性质}
        若$x(t)\leftrightarrow X(f)$,则
        \begin{equation}
            \frac{\dif x(t)}{\dif t}\leftrightarrow (j2\pi f)X(f)
        \end{equation}
        \Proof 由\defref{def:ft}得
        \begin{equation*}
            x(t) =\int_{-\infty}^{\infty}X(f)e^{j2\pi ft}\dif f
        \end{equation*}
        将等式两端对$t$求导,则有
        \begin{equation*}
            \frac{\dif x(t)}{\dif t} =\frac{\dif}{\dif t}\int_{-\infty}^{\infty}X(f)e^{j2\pi ft}\dif f
        \end{equation*}
        交换微分和积分的次序,可得
        \begin{equation*}
            \frac{\dif x(t)}{\dif t} =\int_{-\infty}^{\infty}[(j2\pi f)X(f)]e^{j2\pi ft}\dif f
        \end{equation*}
        对照\defref{def:ft}可知
        \begin{equation*}
            \frac{\dif x(t)}{\dif t} \leftrightarrow (j2\pi f)X(f)
        \end{equation*}

    \paragraph{频域微分性质}
        若$x(t)\leftrightarrow X(f)$,则
        \begin{equation}
            (-j2\pi t)x(t) \leftrightarrow \frac{\dif X(f)}{\dif f}
        \end{equation}
        \Proof 由\defref{def:ft}得
        \begin{equation*}
            X(f) =\int_{-\infty}^{\infty}x(t)e^{-j2\pi ft}\dif t
        \end{equation*}
        将等式两端对$f$求导,则有
        \begin{equation*}
            \frac{\dif X(f)}{\dif f} =\frac{\dif}{\dif f}\int_{-\infty}^{\infty}x(t)e^{-j2\pi ft}\dif t
        \end{equation*}
        交换微分和积分的次序,可得
        \begin{equation*}
            \frac{\dif X(f)}{\dif f} =\int_{-\infty}^{\infty}[(-j2\pi t)x(t)]e^{-j2\pi ft}\dif t
        \end{equation*}
        对照\defref{def:ft}可知
        \begin{equation*}
            (-j2\pi t)x(t) \leftrightarrow \frac{\dif X(f)}{\dif f}
        \end{equation*}
    
    \paragraph{共轭对称性}
        若$x(t)\leftrightarrow X(f)$,则
        \begin{equation}
            x^*(t)\leftrightarrow X^*(-f)
        \end{equation}
        \Proof 
        \begin{equation*}
            \begin{split}
                \mathscr{F}[x^*(t)] &= \int_{-\infty}^{\infty}x^*(t)e^{-j2\pi ft}\dif t\\
                                    &= \left[\int_{-\infty}^{\infty}x(t)e^{-j2\pi (-f)t}\dif t\right]^*\\
                                    &= X^*(-f)
            \end{split}
        \end{equation*}

    \paragraph{内积性质}对于$x(t)\leftrightarrow X(f)$,$y(x)\leftrightarrow Y(x)$
    \begin{equation}\label{eq:inner}
        \int_{-\infty}^{\infty}x(t)y^*(t)\dif t=\int_{-\infty}^{\infty}X(f)Y^*(f)\dif f
    \end{equation} 
    \Proof
    \begin{equation*}
        \begin{split}
            \int_{-\infty}^{\infty}x(t)y^*(t)\dif t &= \int_{-\infty}^{\infty}\left[\int_{-\infty}^{\infty}X(f)e^{j2\pi ft}\dif f\right]y^*(t)\dif t\\
                                                    &= \int_{-\infty}^{\infty}X(f)\left[\int_{-\infty}^{\infty}y^*(t)e^{j2\pi ft}\dif t\right]\dif f\\
                                                    &= \int_{-\infty}^{\infty}X(f)\left[\int_{-\infty}^{\infty}y(t)e^{-j2\pi ft}\dif t\right]^*\dif f\\
                                                    &= \int_{-\infty}^{\infty}X(f)Y^*(f)\dif f
        \end{split}
    \end{equation*}

    \subsection{卷积定理}
    卷积的定义如下
    \begin{mydef}{卷积}\label{def:conv}
        对于$t\in\mathbb{R}$有函数$f(t),g(t)\in\mathbb{C}$
        则规定卷积运算符$*$为
        \begin{equation}
            h(t)=f(t)*g(t)=\int_{-\infty}^{\infty}f(\tau)g(t-\tau)\dif \tau
        \end{equation}
    \end{mydef}
    \subsubsection{时域卷积定理}
    \begin{mythm}{时域卷积定理}\label{thm:tconv}
        若$x_1(t)\leftrightarrow X_1(f),x_2(t)\leftrightarrow X_2(f)$,则
        \begin{equation}
            x_1(t)*x_2(t)\leftrightarrow F_1(f)F_2(f)
        \end{equation}
    \end{mythm}
    \Proof
    \begin{equation*}
        \begin{split}
            \mathscr{F}[x_1(t)*x_2(t)] &=\int_{-\infty}^{\infty}e^{-j2\pi ft}\left[\int_{-\infty}^{\infty}x_1(\tau)x_2(t-\tau)\dif \tau\right]\dif t\\
                                       &=\int_{-\infty}^{\infty}x_1(\tau)\left[\int_{-\infty}^{\infty}e^{-j2\pi ft}x_2(t-\tau)\dif t\right]\dif \tau\\
                                       &=\int_{-\infty}^{\infty}x_1(\tau)e^{-j2\pi f\tau}X_2(f)\dif\tau\\
                                       &=X_1(f)X_2(f)
        \end{split}
    \end{equation*}

    \subsubsection{频域卷积定理}
    \begin{mythm}{频域卷积定理}\label{thm:fconv}
        若$x_1(t)\leftrightarrow X_1(f),x_2(t)\leftrightarrow X_2(f)$,则
        \begin{equation}
            x_1(t)x_2(t)\leftrightarrow F_1(f)*F_2(f)
        \end{equation}
    \end{mythm}
    \Proof
    \begin{equation*}
        \begin{split}
            \mathscr{F}^{-1}[X_1(f)*X_2(f)] &=\int_{-\infty}^{\infty}e^{j2\pi ft}\left[\int_{-\infty}^{\infty}X_1(\mu)X_2(f-\mu)\dif \mu\right]\dif f\\
                                            &=\int_{-\infty}^{\infty}X_1(\mu)\left[\int_{-\infty}^{\infty}e^{j2\pi ft}X_2(f-\mu)\dif f\right]\dif \mu\\
                                            &=\int_{-\infty}^{\infty}X_1(\mu)e^{j2\pi f\mu}x_2(t)\dif\mu\\
                                            &=x_1(t)x_2(t)
        \end{split}
    \end{equation*}

\subsection{帕塞瓦尔定理}
    帕塞瓦尔定理是傅里叶变换的基函数$e^{-j2\pi ft}$正交性的体现
    \begin{mythm}{帕塞瓦尔定理}\label{thm:parseval}
        若$x(t)\leftrightarrow X(f)$,则
        \begin{equation}
            \int_{-\infty}^{\infty}\abs{x(t)}^2\dif t = \int_{-\infty}^{\infty}\abs{X(f)}^2\dif f
        \end{equation}
    \end{mythm}
    \Proof
    \begin{equation*}
        \begin{split}
            \int_{-\infty}^{\infty}\abs{x(t)}^2\dif t &= \int_{-\infty}^{\infty}x(t)x^*(t)\dif t\\
                                                      &= \int_{-\infty}^{\infty}\left[\int_{-\infty}^{\infty}X(f)e^{j2\pi ft}\dif f\right]x^*(t)\dif t\\
                                                      &= \int_{-\infty}^{\infty}X(f)\left[\int_{-\infty}^{\infty}x^*(t)e^{j2\pi ft}\dif t\right]\dif f\\
                                                      &= \int_{-\infty}^{\infty}X(f)\left[\int_{-\infty}^{\infty}x(t)e^{-j2\pi ft}\dif t\right]^*\dif f\\
                                                      &= \int_{-\infty}^{\infty}X(f)X^*(f) \dif f= \int_{-\infty}^{\infty}\abs{X(f)}^2\dif f
        \end{split}
    \end{equation*}

\subsection{典型信号的傅里叶变换}
    \paragraph{矩形脉冲信号与抽样函数} 矩形脉冲信号表达式为
    \begin{equation}
        x(t)=E\text{rect}\left(\frac{t}{\tau}\right)=
        \begin{cases}
            E\phantom{0}\hspace{10pt}\left(\abs{t}<\frac{\tau}{2}\right)\\
            0\phantom{E}\hspace{10pt}\left(\abs{t}>\frac{\tau}{2}\right)\\  
        \end{cases}
    \end{equation}
    \begin{equation*}
        \begin{split}
            X(f)&=\int_{-\infty}^{\infty}x(t)e^{-j2\pi ft}\dif t\\
                &=E\int_{-\frac{\tau}{2}}^{\frac{\tau}{2}}e^{-j2\pi ft}\dif t\\
                &=-\frac{E}{j2\pi f}(e^{-j2\pi f\frac{\tau}{2}}-e^{j2\pi f\frac{\tau}{2}})\\
                &=\frac{E}{\pi f}\sin{2\pi f\frac{\tau}{2}}\\
                &=E\tau \text{sinc}(f\tau)
        \end{split}
    \end{equation*}
    即
    \begin{equation}
        E\text{rect}\left(\frac{t}{\tau}\right)\leftrightarrow E\tau \text{sinc}(f\tau)
    \end{equation}
    相应的,由对称性得到
    \begin{equation}
        2WE\text{sinc}(2Wt)\leftrightarrow E\text{rect}\left(\frac{f}{2W}\right)
    \end{equation}
    \paragraph{直流信号与冲激函数}
    \begin{equation}
        \begin{split}
        \mathscr{F}[\delta(t)] &=\int_{-\infty}^{\infty}\delta(t)e^{-j2\pi ft}\dif t\\
                               &=\int_{-\infty}^{\infty}\delta(t)\dif t\\
                               &=1
        \end{split}
    \end{equation}
    由对称性得到
    \begin{equation}
        \mathscr{F}[1]=\delta (-f)=\delta (f)
    \end{equation}

    \paragraph{符号函数}
        \begin{align*}
            \text{sgn}(t)\leftrightarrow X(f)
        \intertext{应用时域微分性质得}
            2\delta (f) \leftrightarrow (j2\pi f)X(f)
        \end{align*}
        即得
        \begin{equation}
            \mathscr{F}[\text{sgn}(t)]= \frac{1}{j\pi f}
        \end{equation}
        由对称性得
        \begin{equation}\label{eq:hilb}
            \mathscr{F}[\frac{1}{\pi t}]= j\text{sgn}(-t)=-j\text{sgn}(t)
        \end{equation}
    
\subsection{周期冲激序列}
    \subsubsection{一般周期信号的傅里叶变换}
        对于一个周期为$T$的信号,其基频记作$f_0=1/T$
        由\defref{def:fs}逆变换式得一般周期信号的傅里叶级数为
        \begin{equation*}
            s(t) = \sum_{n=-\infty}^{\infty}s_ne^{j\frac{2\pi}{T}nt}
        \end{equation*}
        对上式两边做傅里叶变换得
        \begin{equation*}
            \begin{split}
                S(f)&=\mathscr{F}[s(t)]=\mathscr{F}\left[\sum_{n=-\infty}^{\infty}s_ne^{j\frac{2\pi}{T}nt}\right]\\
                    &=\sum_{n=-\infty}^{\infty}s_n\mathscr{F}[e^{j\frac{2\pi}{T}nt}]\\
                    &=\sum_{n=-\infty}^{\infty}s_n\delta(f-nf_0)
            \end{split}
        \end{equation*}
        其中
        \begin{equation*}
            s_n  = \frac{1}{T}\int_{-\frac{T}{2}}^{\frac{T}{2}}s(t)e^{-j\frac{2\pi}{T}nt}\dif t
        \end{equation*}
        可以知道,一个信号以周期为$T$周期化之后,
        频域相当于原信号频谱以$f_0$间隔用$\delta(f)$进行理想抽样。
        抽样后的冲激函数强度恰等于谱系数$s_n$。

    \subsubsection{周期冲击序列的傅里叶变换}
        取原信号为$\delta(t)$,其傅里叶变换为$\mathscr{F}[\delta(t)]=1$。
        将该信号一周期$T$进行周期化,记作
        \begin{equation}
            \Delta_T(t)=\sum_{n=-\infty}^{\infty}\delta(t-nT)
        \end{equation}
        根据上一节的推导
        \begin{equation*}
            s_n=\frac{1}{T}\int_{-\frac{T}{2}}^{\frac{T}{2}}\delta(t)e^{-j\frac{2\pi}{T}nt}\dif t =\frac{1}{T}
        \end{equation*}
        故有
        \begin{equation}
            \mathscr{F}[\Delta_T(t)]=\frac{1}{T}\sum_{n=-\infty}^{\infty}\delta(f-nf_0)
        \end{equation}
    
    \subsubsection{理想采样}
        对信号$x(t)$在时域进行理想采样,频域体现为周期性搬移:
        \begin{equation}\label{eq:FourierofDelta}
            \begin{split}
                \mathscr{F}[x(t)\sum_{n=-\infty}^{\infty}\delta(t-nT)]&= X(f)*\frac{1}{T}\sum_{n=-\infty}^{\infty}\delta(f-nf_0)\\
                                                                      &= \frac{1}{T}\sum_{n=-\infty}^{\infty}X(f-nf_0)
            \end{split}
        \end{equation}
        对频谱$X(f)$在频域进行理想采样,时域体现为周期性搬移:
        \begin{equation}
            \begin{split}
                \mathscr{F}^{-1}[X(f)\sum_{n=-\infty}^{\infty}\delta(f-nf_0)] &= x(t)*T\sum_{n=-\infty}^{\infty}\delta(t-nT)\\
                                                                              &= T\sum_{n=-\infty}^{\infty}x(t-nT)
            \end{split}
        \end{equation}

\subsection{离散时间傅里叶变换及连续系统的时域仿真}
    \subsubsection{离散时间傅里叶变换}
        在计算机中,不能存储连续信号每一时刻的所有数据,
        所以为了方便模拟,便对连续信号进行抽样。
        在满足\emph{奈奎斯特采样定理}的情况下进行采样,
        由\eqaref{eq:FourierofDelta}表明:
        信号经过理想抽样后的频谱仅在幅度上放大了$1/T$倍并进行了周期化。
        \begin{mydef}{离散时间傅里叶变换(DTFT)}
            对于满足绝对可和连续序列$x(n)$有
            \begin{equation}
                X(e^{j\omega})=\sum_{n=-\infty}^{\infty}x(n)e^{-jn\omega}
            \end{equation}
        \end{mydef}
        不难知道,如果$x(n)$是$x_a(t)$的抽样的话,即$x(n)=x_a(nT)$,
        则有
        \begin{equation*}
            X(e^{j\omega})=\frac{1}{T}\sum_{n=-\infty}^{\infty}X(\Omega-n\Omega_0)
        \end{equation*}
        其中采用另一种傅里叶变换的形式:
        \begin{equation}
            \begin{split}
                x(t)&=\frac{1}{2\pi}\int_{-\infty}^{\infty}X(\Omega)e^{j\Omega t}\dif\Omega\\
                X(\Omega)&=\int_{-\infty}^{\infty}x(t)e^{-j\Omega t}\dif t
            \end{split}
        \end{equation}

    \subsubsection{傅里叶变换的数值计算}
    由上一节的内容可以知道,若想数值计算一个信号的频谱,
    可以先对其以充分小\footnote{严格满足或较好地满足奈奎斯特采样定理}的$T$进行采样,
    对采样后的结果乘$T$倍做DTFT,即近似得到信号的频谱的周期化结果。
    考虑到DTFT的结果为连续值,不方便存储和显示,
    可以使用快速傅里叶变换(FFT)进行计算,得到其频谱的采样\footnote{具体阐述可以见\url{https://zhuanlan.zhihu.com/p/136812943}}。
    
    \subsubsection{卷积的仿真}
    以下不加证明地给出结论。
    对于严格带限或近似带限的连续信号$x_a(t),h_a(t),y_a(t)$,
    若有
    \begin{equation*}
        y_a(t)=x_a(t)*h_a(t)
    \end{equation*}
    在以充分小的间隔$T$采样后,得到三个序列
    $x(n)=x_a(nT),h(n)=h_a(nT),y=y_a(nT)$
    则有
    \begin{equation*}
        y(n)=x(n)\otimes h(n)
    \end{equation*}
    其中$\otimes$为离散卷积,可以使用循环卷积实现,进而使用快速傅里叶变换进行优化。
