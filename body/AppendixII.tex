%AppendixII.tex
\section{希尔伯特变换相关性质}\label{appendix:II}
\subsection{希尔伯特变换的定义}
    希尔伯特变换(Hilbert Transform)是一种积分变换,定义如下
    \begin{mydef}{希尔伯特变换}\label{def:Hilbert}
        实函数$f(t)$的希尔伯特变换及其逆变换定义为
        \begin{equation}
            \begin{split}
                \hat{x}(t)=\mathscr{H}[x(t)]&=\frac{1}{\pi}\int_{-\infty}^{\infty}\frac{x(\tau)}{t-\tau}\dif \tau\\
                x(t)=\mathscr{H}^{-1}[\hat{x}(t)]&=-\frac{1}{\pi}\int_{-\infty}^{\infty}\frac{\hat{x}(\tau)}{t-\tau}\dif \tau
            \end{split}
        \end{equation}
    \end{mydef}
    根据\defref{def:conv},\defref{def:Hilbert}可以写成如下形式
    \begin{equation}
        \begin{split}
            \hat{x}(t)&=x(t)*\frac{1}{\pi t}\\
            x(t)&=\hat{x}(t)*\left(-\frac{1}{\pi t}\right)
        \end{split}
    \end{equation}
    可知希尔伯特变换相当于通过一个单位冲激响应为$\dfrac{1}{\pi t}$的相移全通滤波器

    由\eqaref{eq:hilb}知,该滤波器系统函数为
    \begin{equation}
        H(f)=-j\text{sgn}(f)
    \end{equation}

\subsection{因果信号}
    \emph{因果信号(系统)的频谱(网络函数)的实部和虚部是一对希尔伯特变换对。}

    \Proof
    因果信号$h(t)$满足
    \begin{equation}
        h(t)=h(t)\cdot u(t)
    \end{equation}
    两侧做傅里叶变换得
    \begin{equation*}
        \begin{split}
                  H(f)&=H(f)*\left[\frac{1}{2}\delta(f)+\frac{1}{j2\pi f}\right]\\
            R(f)+jX(f)&=[R(f)+jX(f)]*\left[\frac{1}{2}\delta(f)+\frac{1}{j2\pi f}\right]\\
        \end{split}
    \end{equation*}
    \begin{equation*}
        \begin{split}
            R(f)+jX(f)&=\frac{1}{2}R(f)+R(f)*\frac{1}{j2\pi f}+\frac{j}{2}X(f)+X(f)*\frac{1}{2\pi f}\\
            R(f)+jX(f)&=X(f)*\frac{1}{\pi f}+jR(f)*\left(-\frac{1}{\pi f}\right)
        \end{split}
    \end{equation*}
    对比得
    \begin{equation*}
        \begin{split}
            R(f)&=\mathscr{H}[X(f)]\\
            X(f)&=\mathscr{H}^{-1}[R(f)]
        \end{split}
    \end{equation*}

\subsection{解析信号}
    解析信号的形式如下
    \begin{equation}
        z(t)=x(t)+j\hat{x}(t)
    \end{equation}
    性质如下,自证不难:
    \paragraph{}解析信号的实部和虚部是一对希尔伯特变换对。
    
    \paragraph{}解析信号的频谱密度是单边的,只有正频率才有值。

    \paragraph{}解析信号的实部是实值信号,具有双边频谱密度,幅度是解析信号频谱的一半。

    \paragraph{}解析信号$z(t)$的共轭信号$z^*(t)$的频谱也是单边的,只有负频率。

    \paragraph{}解析信号的实部$x(t)$与虚部$\hat{x}(t)$正交且能量相等,$z(t)$的能量是其实部$x(t)$的两倍。

\subsection{希尔伯特变换对}
\begin{table}[H]
    \caption{希尔伯特变换对}
    \centering
    \begin{tabular}{c|c}
        \hline
        \rowcolor{bupt}
        \makebox[16em][c]{\color{white}$x(t)$} & \makebox[16em][c]{\color{white}$\hat{x}(t)$}\\
        \hline
        $m(t)e^{j2\pi f_0t}$ & $-jm(t)e^{j2\pi f_0t}$\\\hline
        $m(t)\cos (2\pi ft)$   & $m(t)\sin (2\pi ft)$\\\hline
        $m(t)\sin (2\pi ft)$   & $-m(t)\cos (2\pi ft)$\\\hline
        $e^{j2\pi f_0t}$ & $-je^{j2\pi f_0t}$\\\hline
        $\cos (2\pi ft)$   & $\sin (2\pi ft)$\\\hline
        $\sin (2\pi ft)$   & $-\cos (2\pi ft)$\\
        \hline
    \end{tabular}
\end{table}
