%ChapterI
\section{绪论}\label{chapter:I}
\HyperBack{chapter:I}
\subsection{引言}
    本章为《通信原理》的绪论,以介绍课程内容为主。
    本笔记与北京邮电大学出版社的《通信原理(第四版)》(周炯槃等著)相配。
    若无特殊说明,本笔记中所提及的"书中页码"均指如上这本书。
    \begin{figure}[H]
        \begin{equation*}
            \xymatrix{
                *+<1.5em>[F]\txt{信源}\ar[r] & *+<1.5em>[F]\txt{编码}\ar[r] & *+<1.5em>[F]\txt{信道}\ar[r] & *+<1.5em>[F]\txt{译码}\ar[r] & *+<1.5em>[F]\txt{信宿}
            }
        \end{equation*}
        \caption{通信系统模型}
    \end{figure}
    《通信原理I》的主要内容以数字信号和模拟信号的调制与解调为主,
    《通信原理II》则涉及到关于信道与信道编码等问题。

\subsection{关于本笔记}
    本笔记是我自己课后整理记录一些重要的或驳杂的知识点,以备复习使用的。
    全文及代码不同步发行于\href{https://github.com/phydx0803/BuptPoCNoteBook/}{GitHub}。
    %\textcolor{white}{phy喜欢欣洁小姐姐}

    文中的粉色的为超链接,点击转跳。而蓝色的公式、图标编号则为交叉引用,点击可以转跳。
    页眉上的\hspace{0.2em}\firstpageofchapter\hspace{0.2em} 代表返回本章首页,\hspace{0.2em}\firstpageofcontent\hspace{0.2em} 则直接返回目录首页,均为点击转跳。

    有问题请联系笔者,
    直接发QQ\footnote{1310541573}也可以。
    本人也提供一定量的\textcolor{bupt}{\LaTeX{}}论文排版以及\textcolor{bupt}{\LaTeX{}}基础教学。

    封面上的日期是由
    \begin{lstlisting}[language=TeX]
\fcolorbox{bupt}{bupt}{\textbf{\textcolor{white}\today}}
    \end{lstlisting}
    生成,保证与编译日期一致。
    
