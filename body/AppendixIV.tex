%AppendixIV.tex
\section{相关函数公式总结}\label{appendix:IV}
\subsection{能量信号}
    \paragraph{互能量谱密度}
    \begin{equation}
        \begin{split}
            E_{xy}(f)=X(f)Y^*(f)\\
            E_{yx}(f)=Y(f)X^*(f)
        \end{split}
    \end{equation}
    \paragraph{自能量谱密度}
    \begin{equation}
        E_x(f)=X(f)X^*(f)=\abs{X(f)}^2
    \end{equation}
    \paragraph{能量}
    \begin{align}
        E_{xy}=\int_{-\infty}^{\infty}E_{xy}(f)\dif f=\int_{-\infty}^{\infty}x(t)y^*(t)\dif t\\
        E_{yx}=\int_{-\infty}^{\infty}E_{yx}(f)\dif f=\int_{-\infty}^{\infty}y(t)x^*(t)\dif t\\
        E_x=\int_{-\infty}^{\infty}E_x(f)\dif f=\int_{-\infty}^{\infty}x(t)x^*(t)\dif t
    \end{align}
    \paragraph{互相关函数}
    \begin{equation}
        \begin{split}
            R_{xy}(\tau)=\int_{-\infty}^{\infty}x(t+\tau)y^*(t)\dif t\\
            R_{yx}(\tau)=\int_{-\infty}^{\infty}y(t+\tau)x^*(t)\dif t
        \end{split}
    \end{equation}
    \paragraph{自相关函数}
    \begin{equation}
        R_x(\tau)=\int_{-\infty}^{\infty}x(t+\tau)x^*(t)\dif t
    \end{equation}
    \paragraph{对称性}
    \begin{align}
        R_{yx}(\tau)=R_{xy}^*(-\tau)\\
        R_x(\tau)=R_x^*(-\tau)
    \end{align}
    \paragraph{傅里叶变换对}
    \begin{align}
        R_{xy}(\tau)\xleftrightarrow[]{\mathscr{F}}E_{xy}(f)\\
        R_{yx}(\tau)\xleftrightarrow[]{\mathscr{F}}E_{yx}(f)\\
        R_x(\tau)\xleftrightarrow[]{\mathscr{F}}E_x(f)
    \end{align}
    \paragraph{施瓦茨不等式}
    \begin{equation}
        R_x(\tau)\leq E_x\hspace{2em}\abs{R_{xy}(\tau)}^2\leq E_x\cdot E_y
    \end{equation}
    \paragraph{加和公式}
    \begin{align}
        E_{x+y}(f)=E_x(f)+E_y(f)+E_{xy}(f)+E_{yx}(f)\\
        R_{x+y}(f)=R_x(f)+R_y(f)+R_{xy}(f)+R_{yx}(f)\\
    \end{align}
    \paragraph{在$\tau=0$处的值}
    \begin{equation}
        R_x(0)=E_x\hspace{2em}R_{xy}(0)=E_{xy}\hspace{2em}R_{yx}(0)=E_{yx}
    \end{equation}

    \vspace{3ex}
    
\subsection{功率信号}
    截短函数$x_T(t)=\text{rect}(\dfrac{t}{T})x(t)\xleftrightarrow[]{\mathscr{F}}X_T(f)$
    \paragraph{互功率谱密度}
    \begin{equation}
        \begin{split}
            P_{xy}(f)=\lim_{T\to\infty}\left\{\frac{1}{T}X_T(f)Y_T^*(f)\right\}\\
            P_{xy}(f)=\lim_{T\to\infty}\left\{\frac{1}{T}Y_T(f)X_T^*(f)\right\}\\
        \end{split}
    \end{equation}
    \paragraph{功率谱密度}
    \begin{equation}
        P_x(f)=\lim_{T\to\infty}\left\{\frac{1}{T}\abs{X_T(f)}^2\right\}
    \end{equation}
    \paragraph{功率}
    \begin{align}
        P_{xy}=\int_{-\infty}^{\infty}P_{xy}(f)\dif f=\lim_{T\to\infty}\left[\frac{1}{T}\int_{-\frac{T}{2}}^{\frac{T}{2}}x(t)y^*(t)\dif t\right]\\
        P_{yx}=\int_{-\infty}^{\infty}P_{yx}(f)\dif f=\lim_{T\to\infty}\left[\frac{1}{T}\int_{-\frac{T}{2}}^{\frac{T}{2}}y(t)x^*(t)\dif t\right]\\
        P_x=\int_{-\infty}^{\infty}P_x(f)\dif f=\lim_{T\to\infty}\left[\frac{1}{T}\int_{-\frac{T}{2}}^{\frac{T}{2}}\abs{x(t)}^2\dif t\right]
    \end{align}
    \paragraph{互相关函数}
    \begin{equation}
        \begin{split}
            R_{xy}(\tau)=\lim_{T\to\infty}\left[\frac{1}{T}\int_{-\frac{T}{2}}^{\frac{T}{2}}x(t+\tau)y^*(t)\dif t\right]\\
            R_{yx}(\tau)=\lim_{T\to\infty}\left[\frac{1}{T}\int_{-\frac{T}{2}}^{\frac{T}{2}}y(t+\tau)x^*(t)\dif t\right]
        \end{split}
    \end{equation}
    \paragraph{自相关函数}
    \begin{equation}
        R_x(\tau)=\lim_{T\to\infty}\left[\frac{1}{T}\int_{-\frac{T}{2}}^{\frac{T}{2}}x(t+\tau)x^*(t)\dif t\right]
    \end{equation}
    \paragraph{对称性}
    \begin{align}
        R_{yx}(\tau)=R_{xy}^*(-\tau)\\
        R_x(\tau)=R_x^*(-\tau)
    \end{align}
    \paragraph{傅里叶变换对}
    \begin{align}
        R_{xy}(\tau)\xleftrightarrow[]{\mathscr{F}}P_{xy}(f)\\
        R_{yx}(\tau)\xleftrightarrow[]{\mathscr{F}}P_{yx}(f)\\
        R_x(\tau)\xleftrightarrow[]{\mathscr{F}}P_x(f)
    \end{align}
    \paragraph{施瓦茨不等式}
    \begin{equation}
        R_x(\tau)\leq P_x\hspace{2em}\abs{R_{xy}(\tau)}^2\leq P_x\cdot P_y
    \end{equation}
    \paragraph{加和公式}
    \begin{align}
        P_{x+y}(f)=P_x(f)+P_y(f)+P_{xy}(f)+P_{yx}(f)\\
        R_{x+y}(f)=R_x(f)+R_y(f)+R_{xy}(f)+R_{yx}(f)\\
    \end{align}
    \paragraph{在$\tau=0$处的值}
    \begin{equation}
        R_x(0)=P_x\hspace{2em}R_{xy}(0)=P_{xy}\hspace{2em}R_{yx}(0)=P_{yx}
    \end{equation}